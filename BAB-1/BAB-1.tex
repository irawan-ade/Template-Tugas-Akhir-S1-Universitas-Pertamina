%\renewcommand{\thefootnote}{\arabic{footnote}}
\chapter{PENDAHULUAN}
\label{BAB1:pendahuluan}

\section{Latar Belakang}
 Dokumen ini menggunakan \textit{style} yang sudah didefinisikan di folder Etc/skripsityle.cls. Jangan ubah kode yang ada di \textit{file} tersebut agar dokumen ini tetap memenuhi Pedoman Laporan Tugas Akhir Universitas Pertamina. \textit{File} utama dari dokumen ini adalah \verb|main.tex|. Isi informasi yang ada di \textit{file} tersebut. Tulisan bab x dapat diisi dalam file \verb|BAB-x.tex| yang berada di dalam \textit{folder}-nya masing-masing. Jika ada gambar yang akan dimasukkan dalam suatu bab, maka gambar tersebut dalam dimasukkan dalam \textit{folder} di bab tersebut berada. Perintah \verb|\lipsum[]| digunakan untuk menghasilkan kalimat-kalimat buatan. Silahkan hapus perintah tersebut dari \textit{template}. Simbol \verb|%| digunakan untuk membuat \textit{comment}, yaitu kode yang tidak dieksekusi oleh compiler \LaTeX, sehingga dapat digunakan untuk membuat catatan atau membatalkan suatu perintah.

 Untuk membuat paragraf baru cukup pisahkan satu baris antar paragraf. Tentukan singkatan dan akronim untuk pertama kalinya digunakan dalam teks, walaupun telah didefinisikan dalam abstrak. Tulis definisinya atau nama panjangnya terlebih dahulu sebelum akronimnya. Sebagai contoh, Fakultas Sains dan Ilmu Komputer (FSIK). Singkatan yang umum seperti PERTAMINA, IEEE, ac, dc, dan rms tidak harus didefinisikan. Jangan gunakan singkatan dalam judul kecuali tidak dapat dihindari.

 Untuk membuat persamaan matematika, dapat mengacu ke tulisan berikut: \url{https://en.wikibooks.org/wiki/LaTeX/Mathematics}. 
  
\section{Rumusan Masalah}
\lipsum[7] % generate dummy sentences

% \section{Batasan Masalah}
% \lipsum[8] % generate dummy sentences
 
\section{Tujuan Penelitian}
\lipsum[9] % generate dummy sentences

\section{Manfaat Penelitian}
\lipsum[10] % generate dummy sentences